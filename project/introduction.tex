\section*{\fontsize{25}{29}\selectfont{\textbf Introduction} \\}
\fontsize{20}{24}\selectfont{\color{red} {TITLE:  }}
\Large\textbf{{Fault attack on RoadRunneR: A Small and Fast Bit slice Block Cipher for Low Cost 8-Bit Processors.}}

\section*{\fontsize{20}{24}\selectfont{\color{purple} {Motivation:  }}}

\paragraph{}With this growing field of computer Science , the prices of the small electronic devices decreases, leading to a very fine exposure to the notions like ubiquitous computing, Internet of things. The availability and programming nature of these CPU available in the market leads to  acquainted economy and the reach to the loopholes of this technology becomes less vulnerable.

Thus the responsibility of the crypto-analysts increases to figure out the loopholes in the design before the attack begins.

\section*{\fontsize{20}{24}\selectfont{\color{purple} {Problem Statement:  }}}

Now with this insight we can do fault attack on RoadRunneR block cipher and after taking both cipher text and faulty cipher text (that comes after putting fault in the 9th round key)we try to figure out the key of block cipher.

Basically we get the equations from the cipher(i.e in figure 1) in which there are some unknown variables and we try to find that. \\


